Si consideri il sistema lineare
\begin{equation*}
A x = b
\end{equation*}
descritto dalla matrice tridiagonale $A$:
\begin{equation*}
\mathbb{R}^{(M)\times (M)}\ni A\eqbydef\left[
\begin{array}{ccccccc}
4       & -1        & 0         & \ldots    & 0         & 0     \\
-1      & 4         & -1        & \ldots    & 0         & 0     \\
\vdots  &           & \ddots    &           &           &       \\
0       & 0         & \ldots    & -1        & 4         & -1    \\
0       & 0         & \ldots    & 0         & -1        & 4     \\
\end{array}
\right].
\end{equation*}
e con
\begin{equation*}
\mathbb{R}^{(M)}\ni b\eqbydef\left[
\begin{array}{c}
1 \\
1 \\
\vdots \\
1
\end{array}
\right].
\end{equation*}

Per risolvere questo sistema � possibile ricorrere al metodo del
gradiente coniugato precondizionato. La libreria \cpp{C++} IML++
fornisce una raccolta di algoritmi iterativi per la soluzione di sistemi
lineari, scritti in modo da essere indipendenti dalla implementazione
delle strutture dati (matrici e vettori): si sfrutti la funzione
\cpp{template} \cpp{cg} fornita nel file \texttt{cg.h}, definendo una
classe \cpp{Preconditioner} da fornire come argomento \cpp{template}.
\lstset{basicstyle=\scriptsize\sf}
\lstinputlisting[caption=Interfaccia della funzione \cpp{template}
\cpp{cg},linerange={26-29}]{./es/src/cg.hpp}
\vspace{.5cm}
\lstset{basicstyle=\sf}

Nei files \texttt{matrix16.mat} e \texttt{matrix400} vengono fornite
(sotto forma di elenco degli elementi non nulli)
due possibili scelte per la matrice $A$,
rispettivamente di dimensioni $16\times 16$ e $400\times 400$: si
riporta qui l'incipit del file \texttt{matrix16.mat}.
\lstset{basicstyle=\scriptsize\sf}
\lstinputlisting[caption=File \texttt{matrix16.mat} contenente la descrizione della matrice
$A$,linerange={1-10}]{./es/matrix16.mat}
\lstset{basicstyle=\sf}
La prima sezione del file � definita dall'intestazione \cpp{# SIZE}
e contiene le dimensioni della matrice; la prima riga dopo
l'intestazione \cpp{# NZERO} indica il numero di elementi non nulli
della matrice, che vengono di seguito elencati.

Si scriva un codice in grado di costruire la matrice leggendola da
file: quindi si risolva il sistema lineare con il metodo del gradiente
coniugato precondizionato, con tolleranza $10^{-10}$ e massimo numero
di iterazioni $1000$. Per la memorizzazione della matrice si utilizzino il formato denso
ed il formato sparso e si confrontino le prestazioni del codice
realizzato nei due casi (occupazione di memoria, tempo di esecuzione).
\vspace{.5cm}

\textit{Suggerimento:} si utilizzino le definizioni di matrici e
vettori fornite dalla libreria \cpp{boost::numeric::ublas}. Questa
stessa libreria fornisce anche un solutore per sistemi lineari a
matrice triangolare che pu� essere sfruttato per la costruzione di un
semplice precondizionatore triangolare.

Per valutare le prestazioni del codice si utilizzino gli strumenti
\texttt{valgrind} e \texttt{gprof}.

