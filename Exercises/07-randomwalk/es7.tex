\documentclass[10pt,a4paper]{article}
\usepackage[T1]{fontenc}
\usepackage[a4paper]{geometry}
\usepackage{xcolor}
\usepackage{amssymb}
\usepackage{amsmath}
\usepackage{graphicx}
\usepackage{tabularx}
\usepackage{multirow}
\usepackage{subfigure}
\usepackage{verbatim}
\usepackage{fancyhdr}
\usepackage{listings}
\usepackage{../common/espacs}

%\input{../common/commands.tex}

\newcommand*{\vect}[1]{\boldsymbol{#1}}
\newcommand*{\mat}[1]{\boldsymbol{#1}}

\title{Exercise Session 5}
\date{November 9, 2012}

\pagestyle{fancy}
\headheight 35pt

\begin{document}
\lstset{language=[ISO]C++}
\maketitle

\section*{Data mining: $k$-means clustering}

$k$-means clustering is a common method used in data mining, which aims to
partition $N$ objects into $k$ clusters, with $k << N$. The clusters are built
such as each object belongs to the \emph{nearest} cluster, so there must be a
notion of distance between objects.

An object example can be a point in $\RR^n$ equipped with the Euclidean norm. In
this case the distance would be computed with respect to the centroid of the
cluster and the space would be partitioned in Voronoi cells. The approach is
however more genaral, it can be applied to any set of objcts with a kwnoledge of
a distance between then, e.g. functions in a suitable space and a suitable norm,
etc.

%%%%%%%%%%%%%%
%\begin{figure}[htb]
%\centering
%$\vcenter{\hbox{\includegraphics[width=.3\textwidth]{fig/wiki1}}}$
%\hfil
%$\vcenter{\hbox{\includegraphics[width=.3\textwidth]{fig/wiki2}}}$
%\\[1ex]
%$\vcenter{\hbox{\includegraphics[width=.3\textwidth]{fig/wiki3}}}$
%\hfil
%$\vcenter{\hbox{\includegraphics[width=.3\textwidth]{fig/wiki4}}}$
%\caption{$k$-means algorithm in $\RR^2$. The squares represent the objects to be
%clustered, while the circles represent the centroids of the partitions. Each
%}
%\label{fig:alg}
%\end{figure}
%%%%%%%%%%%%%%
The problem is NP-hard, but an algorithm can be devised in order to approach a
local optimum. The algorithm is depicted in Fig. \ref{fig:alg} for the case of
points in $\RR^2$. Given a set of initial centroids, the objects are associated
to the nearest centroid to build up the clusters. Afterwords, the centroid is
updated to the point that minimizes the distance between the objects that belong
to the partition. In the case of the Euclidean norm it will be the barycenter of
the points in the cluster. This concludes a step of the algorithm, that is
repeated starting again assigning the points to the nearest centroid.

The algorithm stops when the clusters are fixed: when all the points are again
assigned to the same partition, the update of the centroids gives a null
increment. Note that, since we only reach a local optimum, the solution is
dependent on the choice of the initial guess for the centroids.

\section*{Exercise 1}

Implement a random walk in one dimension, using the following tips:
\begin{itemize}
  \item use a \cpp{map} to store the distribution of particles.
  \item use as starting configuration a Dirac delta --- approximated with
  $10000$ particles located at zero (use a much smaller value until the code
  is functioning properly).
  \item consider that the particles move by 1 or -1 at each time step,
  \item use a \cpp{discrete_distribution} to emulate the random motion, with
  equal probability of moving forwards or backwards.
  \item simulate 20 time steps.
  \ item print out the distribution to screen and to file, in order to plot it
  in \texttt{gnuplot}.
  \item plot also the fundamental solution in gnuplot to compare the result,
  remembering that $D = \frac{h^2}{2t}$.
  \item check what happens if the distrubution is biased towards one direction,
  or if the particle is allowed to remain in the same position, without	moving.
\end{itemize}

\section*{Exercise 2}

Implement a class that performs MonteCarlo (MC) integration, using a suitable
probability distribution.



%\section*{Solution}

\subsection*{Exercise 1}

The solution is spread in different files. The first we look at is
\texttt{Util.hpp}
%
\lstset{basicstyle=\scriptsize\sf}
\lstinputlisting[caption=\texttt{Util.hpp} file.]
    {./src/ex1/Util.hpp}
\lstset{basicstyle=\sf}
%
The \cpp{Rescaler} is a simple utility that maps an integer number in
$[0,\ldots,N)$ in an integer number in $\{-1,0,1\}$. This class will be used to
translate the integer numbers generated by the random engine in	velocities for
the set of particles.

The \cpp{oflag_T} is a \cpp{typedef} that is useful to declare the different way
to output in our routines. The approach that we use here uses bit-flags, and is
simila to the way that input/output flags are managed in the STL. each option is
represented by a single bit inside a wider integer variable. Afterwards, all
possible combinations of options can be built using bit-wise logical operators.
In a similar way also the \cpp{std::bitset} can be used.

The second file we look at is \texttt{Distributions.hpp}
%
\lstset{basicstyle=\scriptsize\sf}
\lstinputlisting[caption=\texttt{Distibutions.hpp} file.]
    {./src/ex1/Distributions.hpp}
\lstset{basicstyle=\sf}
%
This file stores a locally implemented distribution called
\cpp{FakeDistribution}. As the name says, this distribution is in fact
fictitious, since it gives exactly equal percentages to each value in the range,
in fact this distribution is meant as a testing for the \cpp{RandomWalk} class.
In order to use it, it must have an identical inteface to the
\cpp{discrete_distribution} class in the STL, at least for the methods that are
used in the \cpp{RandomWalk} class, in particular it must have a constructor
that takes an \cpp{initializer_list} as an argument. 

The main class \cpp{RandomWalk} is implemented in the \texttt{RandomWalk.hpp} 
file that follows
%
\lstset{basicstyle=\scriptsize\sf}
\lstinputlisting[caption=\texttt{RandomWalk.hpp} file.]
    {./src/ex1/RandomWalk.hpp}
\lstset{basicstyle=\sf}
%
In order to work with different kind of distribution, the class is templated on
the distribution type. The implementation is then straightforward, making use of
the object oriented philosophy of the blocks that build the final product. We
note the use of the output bit flag to decide which sections of code should be
run.

The main file is minimal
%
\lstset{basicstyle=\scriptsize\sf}
\lstinputlisting[caption=\texttt{main\_randomlap.cpp} file.]
    {./src/ex1/main_randomlap.cpp}
\lstset{basicstyle=\sf}
%
and has the version with the \cpp{FakeDistribution} commented. The use of
different weights or a weight also for particles that do not move is achieved by
simply passing the correct initializer list that represents the weights.

\subsection*{Exercise 2}

The implementation of a MonteCarlo method to compute integrals can be seen in
the Examples section.


\end{document}
