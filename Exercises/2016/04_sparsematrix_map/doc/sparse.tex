 %&latex
\documentclass[smaller,a4paper]{beamer}
\usepackage{amsmath,amssymb,pdfsync,listings}
\usepackage{verbatim}
\usepackage{graphicx}
\usepackage{truncate}
%%\usepackage{mpmulti}
\usepackage{times}
\lstset{
  language=[ISO]C++,                        % The default language
  basicstyle=\scriptsize,                   % The basic style
  %backgroundcolor=\color{LightGray},       % Set listing background
  %keywordstyle=\color{DarkBlue}\bfseries,  % Set keyword style
  %commentstyle=\color{DarkGreen}\itshape,  % Set comment style
  %stringstyle=\color{DarkRed},             % Set string constant style
  extendedchars=true                        % Allow extended characters
}
\newcommand{\cpp}[1]{\lstinline!#1!}
\usepackage[english]{babel}

\begin{document}
\title{Using STL Containers to Implement a Sparse Matrix Class}
\frame{\titlepage}

%---------------------------------------------------------------------------------

\begin{frame}[fragile]
\    \frametitle{Sparse Matrices}
\begin{itemize}
\item A sparse (square) matrix is a matrix that has a number of non-zero coefficients proportional to $n$ rather than $n^{2}$
\item It is not convenient to store all entries including zeros
\item It is not convenient to include zero entries when doing algebra with sparse matrices
\end{itemize}
\end{frame}

\begin{frame}
    \frametitle{A Sparse Matrix Class Based on STL Containers}
\begin{itemize}
\item Consider the following class:\\[3mm]
\cpp{class sparse_matrix : public std::vector<std::map<unsigned int, double> >}
\item What happens if we do:\\[3mm]
\cpp{A = sparse_matrix (4); auto x = A[2][2] }
\item What happens if we do:\\[3mm]
\cpp{A = sparse_matrix (4); A[2][2] = 1.0;}
\end{itemize}
\end{frame}

\begin{frame}\frametitle{Exercises}
\begin{itemize}
\item Implement a \cpp{sparse_matrix} class inheriting from \cpp{std::vector<std::map<unsigned int, double> >}
\item Adapt fem1d to use the new \cpp{sparse_matrix} class 
\item Replace the \cpp{sparse_matrix} class with \cpp{Eigen::Sparse}
\end{itemize}

\end{frame}




\end{document}