\documentclass[smaller,a4paper]{beamer}
\usepackage{amsmath,amssymb,pdfsync,listings}
\usepackage{graphicx,xcolor}
\usepackage{times}

\usepackage[english]{babel}

\lstnewenvironment{bashcommand}{\lstset{}}{}
{%
    \lstset{%
        basicstyle=\ttfamily,
        escapechar=~,
        morekeywords={gcc,g++},
        keywordstyle=\bfseries,
        literate={\$}{\$}1}
}{}
\lstnewenvironment{bashoutput}
    {\lstset{basicstyle=\ttfamily}}
    {}
    
    
\begin{document}
\title{Name mangling and \\ linking C objects into C++ code}

\frame{\titlepage}

\begin{frame}\frametitle{example C code}
\lstinputlisting[language=c]{hellofunc.c}
\end{frame}


\begin{frame}\frametitle{header for {\tt hellofunc.h}}
\lstinputlisting[language=c]{hellomake.h}
\begin{itemize}
\item the preprocessor macro \lstinline[language=c]{__cplusplus} 
      is defined by default whenever using a C++ compiler
\item the effect of the \lstinline[language=c]{\#ifdef __cplusplus} conditional
      can be verified by running the preprocessor       
\end{itemize}
\end{frame}

\begin{frame}\frametitle{example C++ code}
\lstinputlisting[language=c]{hellomake.cpp}
\end{frame}


\begin{frame}[fragile]\frametitle{invoking the proprocessor}

using \lstinline[language=bash]{g++} the lines between conditionals
are included in the output of the preprocessor 

\begin{bashcommand}
$ g++ -E hellomake.h
\end{bashcommand}

\begin{bashoutput}
# 1 "hellomake.h"
# 1 "<built-in>"
# 1 "<command-line>"
# 1 "hellomake.h"
extern "C"{
void myPrintHelloMake (void);
}
\end{bashoutput}

using \lstinline[language=bash]{gcc} the lines between conditionals
are \emph{not} included in the output of the preprocessor 

\begin{bashcommand}
$ gcc -E hellomake.h
\end{bashcommand}

\begin{bashoutput}
# 1 "hellomake.h"
# 1 "<built-in>"
# 1 "<command-line>"
# 1 "hellomake.h"
void myPrintHelloMake (void);
\end{bashoutput}

\end{frame}

\begin{frame}[fragile]\frametitle{C vs. C++ linkage}

Why is the \lstinline[language=C++]|extern "C"{...}| needed?

\null

Let us compile \lstinline{hellofunc.c} using the C compiler
and inspect the symbols in the object file
\begin{bashcommand}
$ gcc -c hellofunc.c
$ nm hellofunc.o | grep myPrintHelloMake
\end{bashcommand}
\begin{bashoutput}
0000000000000000 T _myPrintHelloMake
\end{bashoutput}
The name of the compiled function in the object file is (almost) the same
as defined in the C source code.
\end{frame}

\begin{frame}[fragile]\frametitle{C vs. C++ linkage}

Why is the \lstinline[language=C++]|extern "C"{...}| needed?

\null

Let us compile \lstinline{hellofunc.c} using the C++ compiler
and inspect the symbols in the object file
\begin{bashcommand}
$ g++ -c hellofunc.c
$ nm hellofunc.o | grep myPrintHelloMake
\end{bashcommand}
\begin{bashoutput}
0000000000000000 T __Z16myPrintHelloMakev
\end{bashoutput}
The name of the compiled function in the object file is surrounded
by additional unreadable charecters. 

\null 

why?
\end{frame}


\begin{frame}[fragile]\frametitle{C++ name mangling}
Functions in C++ are distinguished not only by their name, 
but also by the type of their input parameters.

\null

The C++ compiler encodes the full signature of the function in the
name stored in the object file, this is called \emph{name mangling}.

\null

The name can be \emph{demangled} using the \lstinline[language=bash]|c++filt| utility
\begin{bashcommand}
$ g++ -c hellofunc.c
$ nm hellofunc.o | grep myPrintHelloMake | c++filt
\end{bashcommand}
\begin{bashoutput}
0000000000000000 T myPrintHelloMake()
\end{bashoutput}
The information that the function requires no input parameters is also stored in the object file!

\null

When invoking a C function in C++ code, the compiler must be told to search for a symbol that has
C linkage and is given an unmangled label in the object file, this is the purpose of the \lstinline[language=C++]|extern "C"{...}|
\end{frame}

\begin{frame}[fragile]\frametitle{what happens specifying C linkage?}
Let us run the compiler on \lstinline[language=bash]|hellomake.cpp|

\begin{bashcommand}
$ g++ -c hellomake.cpp -I.
$ nm hellomake.o | grep myPrintHelloMake
\end{bashcommand}
\begin{bashoutput}
                 U _myPrintHelloMake
\end{bashoutput}
The object file has an \emph{undefined symbol} with the \emph{unmangled name}.
This means the linker expects the symbol with the \emph{unmangled name} to exist
when linking the final binary. 
If we run the C compiler on  \lstinline[language=bash]|hellofunc.c| this 
symbol will be defined in \lstinline[language=bash]|hellofunc.o| so we can link the two object files and form an executable
\begin{bashcommand}
$ gcc -c hellofunc.c
$ g++ -c hellomake.cpp -I.
$ g++ hellomake.o hellofunc.o -o hellomake
$ ./hellomake 
\end{bashcommand}
\begin{bashoutput}
Hello makefiles!
Hello from C++ as well!
\end{bashoutput}
\end{frame}


\begin{frame}[fragile]\frametitle{what happens without specifying C linkage?}
Let us comment out the lines specifying C linkage in the \lstinline[language=bash]|hellomake.h|
header 
\begin{lstlisting}[language=C++]
//#ifdef __cplusplus
//extern "C"{
//#endif
void myPrintHelloMake (void);
//#ifdef __cplusplus
//}
//#endif
\end{lstlisting}

and run the compiler on \lstinline[language=bash]|hellomake.cpp|

\begin{bashcommand}
$ g++ -c hellomake.cpp -I.
$ nm hellomake.o | grep myPrintHelloMake
\end{bashcommand}
\begin{bashoutput}
                 U __Z16myPrintHelloMakev
\end{bashoutput}
The object file has an \emph{undefined symbol} with the \emph{mangled name}.
This means the linker expects the symbol with the \emph{mangled name} to exist
when linking the final binary, and will error out if it isn't!

\end{frame}

\begin{frame}[fragile]\frametitle{what happens without specifying C linkage?}
Let us try to link the executable
\begin{bashcommand}
$ gcc -c hellofunc.c
$ g++ -c hellomake.cpp -I.
$ g++ hellofunc.o hellomake.o -o hellomake
\end{bashcommand}
\begin{bashoutput}
Undefined symbols for architecture x86_64:
  "myPrintHelloMake()", referenced from:
      _main in hellomake.o
ld: symbol(s) not found for architecture x86_64
collect2: error: ld returned 1 exit status
\end{bashoutput}
\end{frame}

\begin{frame}[fragile]\frametitle{what happens without specifying C linkage?}
If I had compiled the C file with the C++ compiler linking would have worked
\begin{bashcommand}
$ g++ -c hellofunc.c
$ g++ -c hellomake.cpp -I.
$ g++ hellofunc.o hellomake.o -o hellomake
$ ./hellomake 
\end{bashcommand}
\begin{bashoutput}
Hello makefiles!
Hello from C++ as well!
\end{bashoutput}
This cannot always be done (especially if linking to a C library that is not compiled by yourself!)
\end{frame}

\begin{frame}\frametitle{example F77 code}
It is (unfortunaley) still common to find numerical libraries written in fortran, 
so let us consider a simple fortran 77 subroutine
\lstinputlisting[language=fortran]{hellofunc.f77}
\end{frame}

\begin{frame}[fragile]\frametitle{linking fortran objects}
The situation when linking to fortran77 objects is similar, but has some differnces
\begin{bashcommand}
$ nm hellofunc.o | grep -i myPrintHelloMake 
$ gfortran -c hellofunc.f
\end{bashcommand}
\begin{bashoutput}
0000000000000000 T _myprinthellomake_
\end{bashoutput}
\begin{itemize}
\item fortran is case insensitive
\item an additional underscore has been added to the object name (this depends on the compiler) 
\end{itemize}
To use the f77 subroutine in a C++ program we need to define in the header
\begin{lstlisting}[language=C++]
extern "C" { void myprinthellomake_ (void); }
\end{lstlisting}
\end{frame}
\end{document}