 %&latex
\documentclass[smaller,a4paper]{beamer}
\usepackage{amsmath,amssymb,pdfsync,listings}
\usepackage{verbatim}
\usepackage{graphicx}
\usepackage{truncate}
%%\usepackage{mpmulti}
\usepackage{times}
\lstset{
  language=[ISO]C++,                       % The default language
  basicstyle=\scriptsize,                  % The basic style
  %backgroundcolor=\color{LightGray},       % Set listing background
  %keywordstyle=\color{DarkBlue}\bfseries,  % Set keyword style
  %commentstyle=\color{DarkGreen}\itshape,  % Set comment style
  %stringstyle=\color{DarkRed},             % Set string constant style
  extendedchars=true                       % Allow extended characters
}
\newcommand{\cpp}[1]{\lstinline!#1!}
\usepackage[english]{babel}

\begin{document}
\title{The Eigen C++ template library for linear algebra}
\frame{\titlepage}

%---------------------------------------------------------------------------------

\begin{frame}

    \frametitle{Eigen}

    \begin{block}{Introduction to}
        \centering
        Eigen 3
    \end{block}

    \begin{figure}
        \centering
        \includegraphics[width=0.2\textwidth]{images/eigen-logo}
    \end{figure}

    \begin{block}{Useful web links}
        \centering
        \begin{itemize}
            \item \url{http://eigen.tuxfamily.org}
        \end{itemize}
    \end{block}

\end{frame}

%---------------------------------------------------------------------------------

\begin{frame}

    \frametitle{Eigen}

    \begin{block}{Eigen}
        The Eigen library is a linear algebra library that works with dense and
        sparse matrices. With the use of advanced programming techniques it
        performs algebraic operations with very high efficiency. It provides
        algorithms for the solution of linear systems, factorizations, and also
        interfaces to external libraries that perform linear algebra operations.
    \end{block}

    \vspace{1cm}

    \begin{block}{}
        It is widely used as a library in many applications, i.e. at Google.
    \end{block}

\end{frame}

%---------------------------------------------------------------------------------

\begin{frame}[fragile]

    \frametitle{Eigen}

    \begin{block}{Introduction}
        \begin{lstlisting}
#include <iostream>
#include <Eigen/Dense>
int main()
{
 Eigen::MatrixXd m(2,2);
 m(0,0) = 3;
 m(1,0) = 2.5;
 m(0,1) = -1;
 m(1,1) = m(1,0) + m(0,1);
 std::cout << m << std::endl;
}
        \end{lstlisting}
    \end{block}

\end{frame}

%---------------------------------------------------------------------------------

\begin{frame}[fragile]

    \frametitle{Eigen}

        \begin{block}{(Dense) Matrices}
            Generic matrix
            \begin{lstlisting}
Matrix<typename Scalar, int RowsAtCompileTime,
                        int ColsAtCompileTime>
            \end{lstlisting}
        \end{block}

        \begin{block}{Special cases}
            \begin{itemize}
                \item Matrix with fixed dimensions
                    \begin{lstlisting}
typedef Matrix<float, 4, 4> Matrix4f;
                    \end{lstlisting}
                \item Matrix with non-fixed dimensions
                    \begin{lstlisting}
typedef Matrix<double, Dynamic, Dynamic> MatrixXd;
                    \end{lstlisting}
                \item Matrix with only one fixed dimension
                    \begin{lstlisting}
typedef Matrix<int, Dynamic, 1> VectorXi;
                    \end{lstlisting}
            \end{itemize}
        \end{block}

\end{frame}

%---------------------------------------------------------------------------------

\begin{frame}[fragile]

    \frametitle{Eigen}

    \begin{block}{Initialization}
        \begin{itemize}
            \item non-initialized matrix
            \begin{lstlisting}
MatrixXd m(2,2);
            \end{lstlisting}
            \item initialized matrix
            \begin{lstlisting}
Matrix3f m;
m << 1, 2, 3, 4, 5, 6, 7, 8, 9;
            \end{lstlisting}
            \item initialized matrix
            \begin{lstlisting}
MatrixXf m = MatrixXf::Constant(3, 3, 1.2);
            \end{lstlisting}
        \end{itemize}

    \end{block}

\end{frame}

%---------------------------------------------------------------------------------

\begin{frame}[fragile]

    \frametitle{Eigen}

    \begin{block}{Arithmetics}
        Eigen implements all the arithmetic operators, and also
        \begin{itemize}
            \item transpose \cpp{a.transpose()}
            \item conjugate \cpp{a.conjugate()}
            \item adjoint \cpp{a.adjoint()}
            \item dot product \cpp{a.dot(b)}
            \item cross product \cpp{a.cross(b)}
            \item \ldots
        \end{itemize}

    \end{block}

\end{frame}

%---------------------------------------------------------------------------------

\begin{frame}[fragile]

    \frametitle{Eigen}

    \begin{block}{Expression Templates}
        the code
        \begin{lstlisting}
VectorXf a(50), b(50), c(50), d(50);
...
a = 3*b + 4*c + 5*d;
        \end{lstlisting}
        is implemented in such a way that it is equivalent to the execution of
        \begin{lstlisting}
for(int i = 0; i < 50; ++i)
  a[i] = 3*b[i] + 4*c[i] + 5*d[i];
        \end{lstlisting}
        just like the most performant (and optimizable by the compiler) c code.
    \end{block}

\end{frame}

%---------------------------------------------------------------------------------

\begin{frame}[fragile]

    \frametitle{Eigen}

    \begin{block}{Advantages \& disadvantages}
        the efficiency of the code greatly improves, but expression of the type
        \begin{lstlisting}
a = a.transpose();
        \end{lstlisting}
        are not allowed. Furthermore, compile time errors and execution time
        errors become mostly unreadable.
    \end{block}

\end{frame}

\begin{frame}[fragile]

    \frametitle{Eigen}

    \begin{block}{code}
        \begin{lstlisting}
#include <iostream>
#include <Eigen/Dense>
int main() {
  Eigen::Matrix3f m = Eigen::Matrix3f::Random();
  std::cout << "m =" << std::endl << m ; }
        \end{lstlisting}
    \end{block}

    \begin{block}{compilation}
    Since Eigen is a pure header file library, we need to add its path in the
    compilation phase with the parameter \verb1-I1
    \begin{verbatim}
g++ -I/path/to/eigen/ my_program.cpp -o my_program -O2
    \end{verbatim}
    in the virtual machine eigen is available as a module
    \end{block}

\end{frame}

\begin{frame}[fragile,allowframebreaks]
\frametitle{Column-major and row-major storage}
We say that a matrix is stored in row-major order if it is stored row by row. 
The entire first row is stored first, followed by the entire second row, and so on. Consider for example the matrix. {it E.g.}:

$$
A =
\left[\begin{array}{cccc}
8 & 2 & 2 & 9 \\
9 & 1 & 4 & 4 \\
3 & 5 & 4 & 5
\end{array}\right]
$$

Column major storage
$$ [8\ 9\ 3\ 2\ 1\ 5\ 2\ 4\ 4\ 9\ 4\ 5]$$

Row major storage
$$ [8\ 2\ 2\ 9\ 9\ 1\ 4\ 4\ 3\ 5\ 4\ 5]$$

\newpage

Example:\small
\begin{lstlisting}
Matrix<int, 3, 4, ColMajor> Acolmajor;
Acolmajor << 8, 2, 2, 9,
             9, 1, 4, 4,
             3, 5, 4, 5;
cout << "The matrix A:" << endl;
cout << Acolmajor << endl << endl; 
cout << "In memory (column-major):" << endl;
for (int i = 0; i < Acolmajor.size(); i++)
  cout << *(Acolmajor.data() + i) << "  ";
cout << endl << endl;
Matrix<int, 3, 4, RowMajor> Arowmajor = Acolmajor;
cout << "In memory (row-major):" << endl;
for (int i = 0; i < Arowmajor.size(); i++)
  cout << *(Arowmajor.data() + i) << "  ";
cout << endl;
\end{lstlisting}
\newpage
Output:
\begin{lstlisting}
The matrix A:
8 2 2 9
9 1 4 4
3 5 4 5

In memory (column-major):
8  9  3  2  1  5  2  4  4  9  4  5  

In memory (row-major):
8  2  2  9  9  1  4  4  3  5  4  5
\end{lstlisting}
\end{frame}

\begin{frame}[fragile,allowframebreaks]
\frametitle{Interfacing with raw buffers: the Map class}
Eigen allows to work with "raw" C/C++ arrays using the
 \cpp{Map} class.

A Map object has a type defined by its Eigen equivalent:
\cpp{Map<Matrix<typename Scalar, int RowsAtCompileTime, int ColsAtCompileTime> >}

\null

\emph{Example}:
\begin{lstlisting}
typedef Matrix<float,1,Dynamic> MatrixType;
typedef Map<MatrixType> MapType;
typedef Map<const MatrixType> MapTypeConst;   // a read-only map
const int n_dims = 5;
  
MatrixType m1(n_dims), m2(n_dims);
m1.setRandom();
m2.setRandom();
float *p = &m2(0);  // get the address storing the data for m2
MapType m2map(p,m2.size());   // m2map shares data with m2
MapTypeConst m2mapconst(p,m2.size());  // a read-only accessor for m2
cout << "m1: " << m1 << endl;
cout << "m2: " << m2 << endl;
cout << "Squared euclidean distance: " << (m1-m2).squaredNorm() << endl;
cout << "Squared euclidean distance, using map: " <<
  (m1-m2map).squaredNorm() << endl;
m2map(3) = 7;   // this will change m2, since they share the same array
cout << "Updated m2: " << m2 << endl;
cout << "m2 coefficient 2, constant accessor: " << m2mapconst(2) << endl;
/* m2mapconst(2) = 5; */   // this yields a compile-time error
\end{lstlisting}

\end{frame}

\begin{frame}\frametitle{Exercises}
\begin{itemize}
\item Re-implement the fem1d code using Eigen/Dense\\[5mm]
\begin{itemize}
\item Version 1: use an \cpp{Eigen::Map} object to implement 
\cpp{matrix::solve} and \cpp{matrix operator* (matrix&, matrix&)}\\[5mm]
\item version 2: get rid of the \cpp{matrix} class and use \cpp{Eigen::Matrix} instead.
\end{itemize}
\end{itemize}
\end{frame}

\end{document}