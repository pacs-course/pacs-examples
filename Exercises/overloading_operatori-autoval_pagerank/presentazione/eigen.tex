\documentclass{beamer}
\usepackage{beamerthemesplit,bm}
\usepackage[latin1]{inputenc}
\usepackage[italian]{babel}
\usepackage{graphicx}
\usepackage{hyperref}
\usepackage{multimedia}
\usepackage{subfigure}
\usepackage{xcolor}
\usepackage{amsmath,amssymb}

\usepackage{listings}
% User-defined colors
\definecolor{DarkGreen}{rgb}{0, .5, 0}
\definecolor{DarkBlue}{rgb}{0, 0, .5}
\definecolor{DarkRed}{rgb}{.5, 0, 0}
\definecolor{LightGray}{rgb}{.8, .8, .8}

% Settings for listing class
\lstset{
  language=[ISO]C++,                       % The default language
  basicstyle=\scriptsize,                % The basic style
  backgroundcolor=\color{LightGray},       % Set listing background
  keywordstyle=\color{DarkBlue}\bfseries,  % Set keyword style
  commentstyle=\color{DarkGreen}\itshape,  % Set comment style
  stringstyle=\color{DarkRed},             % Set string constant style
  extendedchars=true                       % Allow extended characters
}

% A command to print C++ instructions
\newcommand{\cpp}[1]{\lstinline!#1!}

\usetheme{Boadilla}


\definecolor{mygreen}{rgb}{0,0.48,0.0}

\definecolor{myblue}{rgb}{0,0,0.64}

\author{Alessio Fumagalli}
\date{3-11-2011}
\institute{Politecnico di Milano}

\begin{document}

%---------------------------------------------------------------------------------

\begin{frame}

    \frametitle{Eigen}

    \begin{block}{Introduzione a}
        \centering
        Eigen 3.0
    \end{block}

    \begin{figure}
        \centering
        \includegraphics[width=0.2\textwidth]{images/eigen-logo}
    \end{figure}

    \begin{block}{Pagine web utili}
        \centering
        \begin{itemize}
            \item http://eigen.tuxfamily.org/index.php?title=Main\_Page
        \end{itemize}
    \end{block}

\end{frame}

%---------------------------------------------------------------------------------

\begin{frame}

    \frametitle{Eigen}

    \begin{block}{Eigen}
        Eigen \`e una libreria di algebra lineare che fornisce l'implementazione di matrici
        piene e matrici sparse. Tramite tecniche di programmazione sofisticate permette di
        effettuare operazioni algebriche con incredibile efficienza. Fornisce direttamente
        algoritmi per la risoluzione di sistemi e fattorizzazioni, oppure interfacce per
        utilizzare i solutori lineari pi\`u comuni.
    \end{block}

    \vspace{1cm}

    \begin{block}{}
        � anche utilizzato come libreria in molte applicazioni, ad esempio Google.
    \end{block}

\end{frame}

%---------------------------------------------------------------------------------

\begin{frame}[fragile]

    \frametitle{Eigen}

    \begin{block}{Introduzione}
        \begin{lstlisting}
#include <iostream>
#include <Eigen/Dense>
int main()
{
 Eigen::MatrixXd m(2,2);
 m(0,0) = 3;
 m(1,0) = 2.5;
 m(0,1) = -1;
 m(1,1) = m(1,0) + m(0,1);
 std::cout << m << std::endl;
}
        \end{lstlisting}
    \end{block}

\end{frame}

%---------------------------------------------------------------------------------

\begin{frame}[fragile]

    \frametitle{Eigen}

        \begin{block}{Matrici}
            Matrice generica
            \begin{lstlisting}
Matrix<typename Scalar, int RowsAtCompileTime,
                        int ColsAtCompileTime>
            \end{lstlisting}
        \end{block}

        \begin{block}{Casi particolari}
            \begin{itemize}
                \item Matrice con dimensioni fisse
                    \begin{lstlisting}
typedef Matrix<float, 4, 4> Matrix4f;
                    \end{lstlisting}
                \item Matrice con dimensioni non-fisse
                    \begin{lstlisting}
typedef Matrix<double, Dynamic, Dynamic> MatrixXd;
                    \end{lstlisting}
                \item Matrice con una dimensione fissa
                    \begin{lstlisting}
typedef Matrix<int, Dynamic, 1> VectorXi;
                    \end{lstlisting}
            \end{itemize}
        \end{block}

\end{frame}

%---------------------------------------------------------------------------------

\begin{frame}[fragile]

    \frametitle{Eigen}

    \begin{block}{Inizializzazioni}
        \begin{itemize}
            \item matrice non inizializzata
            \begin{lstlisting}
MatrixXd m(2,2);
            \end{lstlisting}
            \item matrice inizializzata
            \begin{lstlisting}
Matrix3f m;
m << 1, 2, 3, 4, 5, 6, 7, 8, 9;
            \end{lstlisting}
            \item matrice inizializzata
            \begin{lstlisting}
MatrixXf m = MatrixXf::Constant(3,3,1.2);
            \end{lstlisting}
        \end{itemize}

    \end{block}

\end{frame}

%---------------------------------------------------------------------------------

\begin{frame}[fragile]

    \frametitle{Eigen}

    \begin{block}{Aritmetica}
        Sono implementati gli usuali operatori aritmetici, inoltre \`e possibile utilizzare
        \begin{itemize}
            \item trasposta \cpp{a.transpose()}
            \item coniugata \cpp{a.conjugate()}
            \item aggiunta \cpp{a.adjoint()}
            \item prodotto scalare \cpp{a.dot(b)}
            \item prodotto vettoriale \cpp{a.cross(b)}
            \item \ldots
        \end{itemize}

    \end{block}

\end{frame}

%---------------------------------------------------------------------------------

\begin{frame}[fragile]

    \frametitle{Eigen}

    \begin{block}{Expression Template}
        Il codice
        \begin{lstlisting}
VectorXf a(50), b(50), c(50), d(50);
...
a = 3*b + 4*c + 5*d;
        \end{lstlisting}
        In realt\`a viene eseguito come
        \begin{lstlisting}
for(int i = 0; i < 50; ++i)
  a[i] = 3*b[i] + 4*c[i] + 5*d[i];
        \end{lstlisting}
    \end{block}

\end{frame}

%---------------------------------------------------------------------------------

\begin{frame}[fragile]

    \frametitle{Eigen}

    \begin{block}{Vantaggi \& svantaggi}
        L'efficienza del codice migliora notevolmente, tuttavia non sono ammesse
        espressioni del tipo
        \begin{lstlisting}
a = a.transpose();
        \end{lstlisting}
        inoltre gli errori sia in fase di compilazione che in fase di esecuzione
        sono di difficile lettura.
    \end{block}

\end{frame}

%---------------------------------------------------------------------------------

\begin{frame}[fragile]

    \frametitle{Eigen}

    \begin{block}{Consideriamo il seguente codice}
        \begin{lstlisting}
#include <iostream>
#include <Eigen/Dense>
int main() {
  Eigen::Matrix3f m = Eigen::Matrix3f::Random();
  std::cout << "m =" << std::endl << m ; }
        \end{lstlisting}
    \end{block}

    \begin{block}{Compilare}
        Essendo Eigen una libreria composta esclusivamente da \textit{header file},
        per la compilazione del codice dobbiamo utilizzare l'opzione \verb1-I1
        \begin{verbatim}
g++ -I/path/to/eigen/ my_program.cpp -o my_program -O2
        \end{verbatim}
        Nel nostro caso non serve essendo installato nella directory standard
        \verb1/usr/local/eigen1.
    \end{block}

\end{frame}

\end{document}
