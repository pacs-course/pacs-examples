 %&latex
\documentclass[smaller,a4paper]{beamer}
\usepackage{amsmath,amssymb,pdfsync,listings}
\usepackage{verbatim}
\usepackage{graphicx}
\usepackage{truncate}
%%\usepackage{mpmulti}
\usepackage{times}

\usepackage[english]{babel}

\begin{document}
\title{GetPot Example}
\frame{\titlepage}

\begin{frame}

    \frametitle{GetPot}

    \begin{block}{Intro to}
        \centering
        GetPot
    \end{block}

    \begin{figure}
        \centering
        \includegraphics[width=0.5\textwidth]{images/GetPotLogo2}
    \end{figure}

    \begin{block}{Web page}
        \centering
        \begin{itemize}
            \item http://getpot.sourceforge.net/
        \end{itemize}
    \end{block}

\end{frame}


\begin{frame}

    \frametitle{GetPot}

    \begin{block}{GetPot}
        GetPot \emph{header file only} library,
        to facilitate command line and config file parsing.
        Useful for changing algorithm parameters without recompiling
         etc\ldots
    \end{block}

    \vspace{.2cm}

    \begin{block}{ }
        GetPot provides a class to parse \texttt{argc} and \texttt{argv} in alternative to POSIX standard getopt (in C).
    \end{block}

\end{frame}

%---------------------------------------------------------------------------------

\begin{frame}[fragile]

    \frametitle{GetPot}

    \begin{block}{Passing parameters on the command line directly}
        \lstset{basicstyle=\scriptsize\sf}
            \lstinputlisting{./examples/es1.cpp}
        \lstset{basicstyle=\sf}
    \end{block}

    \vspace{.2cm}

    \begin{block}{ }
    Parameters are read via the \texttt{()} operator for the \texttt{GetPot} class.
    It requires the name of the parameter to be read and the default value.
    The type is deduced from the class of the default value.
    \end{block}

\end{frame}

%---------------------------------------------------------------------------------

\begin{frame}[fragile]

    \frametitle{GetPot}

        \begin{block}{Run}
            \begin{verbatim}
./main a=10. b=70. nint=100
./main nitn=100 b=70. a=10.
./main a=10. b=70.
./main
            \end{verbatim}
        \end{block}

    \begin{block}{ }
    	Sorting of command line arguments is unimportant    
    \end{block}

\end{frame}

%---------------------------------------------------------------------------------

\begin{frame}[fragile]

    \frametitle{GetPot}

    \begin{block}{Configuration files}
        \verbatiminput{./examples/data}
    \end{block}

\end{frame}

%---------------------------------------------------------------------------------

\begin{frame}[fragile]

    \frametitle{GetPot}

    \begin{block}{To parse the file in C++ code}
        \lstset{basicstyle=\scriptsize\sf}
            \lstinputlisting{./examples/es2.cpp}
        \lstset{basicstyle=\sf}
    \end{block}

\end{frame}

%---------------------------------------------------------------------------------

\begin{frame}[fragile]

    \frametitle{GetPot}

    \begin{block}{To parse the file in C++ code}
        \lstset{basicstyle=\scriptsize\sf}
            \lstinputlisting{./examples/es3.cpp}
        \lstset{basicstyle=\sf}
    \end{block}

\end{frame}

%---------------------------------------------------------------------------------

\begin{frame}[fragile]

    \frametitle{GetPot}

        \begin{block}{Esecuzione}
            \begin{verbatim}
./main
            \end{verbatim}
        \end{block}

\end{frame}


%---------------------------------------------------------------------------------

\begin{frame}[fragile]

    \frametitle{GetPot}

    \begin{block}{Perch\'e non entrambi?}
        The name of the config file may be passed on the command line
        
    \end{block}

\end{frame}

%---------------------------------------------------------------------------------

\begin{frame}[fragile]

    \frametitle{GetPot}

    \begin{block}{Command line and config file}
        \lstset{basicstyle=\scriptsize\sf}
            \lstinputlisting{./examples/es4.cpp}
        \lstset{basicstyle=\sf}
    \end{block}

\end{frame}

%---------------------------------------------------------------------------------

\begin{frame}[fragile]

    \begin{block}{Run}
        \begin{verbatim}
./main -f data1
./main --file data1
./main
        \end{verbatim}
    \end{block}

\end{frame}


\end{document}