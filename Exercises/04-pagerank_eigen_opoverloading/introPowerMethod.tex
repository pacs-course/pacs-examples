\subsection*{Metodo delle potenze}

Il \emph{metodo delle potenze} \`e applicabile a matrici in cui
l'autovalore di modulo massimo $\lambda_1$ abbia molteplicit\`a
unitaria e sia ben separato dall'autovalore immediatamente pi\`u
piccolo in modulo. Il generico passo dell'algoritmo \`e riportato di
seguito:
\begin{align*}
    {\bf q}\iter{k} &= \frac{ A{\bf q}\iter{k-1} }{
        \norm{ A{\bf q}\iter{k-1}}_2 },\\
    {\bf \nu}\iter{k} &= {{\bf q}\iter{k}}^T  A {\bf q}\iter{k},\\
    {\bf w}\iter{k} &= \frac{ {{\bf w}\iter{k-1}}^T A }{
        \norm{ {{\bf w}\iter{k-1}}^T A}_2 },
\end{align*}
dove con $\nu\iter{k}$, ${\bf q}\iter{k}$ e ${\bf w}\iter{k}$ si sono
indicate, rispettivamente, le approssimazioni di $\lambda_1$ e degli
autovalori destro e sinistro ${\bf x}_1$ e ${\bf y}_1$ ad esso
associati. Vale la seguente stima, utile per determinare un criterio
d'arresto:
\begin{align*}
    \module{ \lambda_1 - \nu\iter{k} } \approx %
    \frac{ \norm{ {\bf r}\iter{k} }_2 }{
    \module{ {{\bf w}\iter{k}}^T \cdot{\bf q}\iter{k} } },
\end{align*}
dove ${\bf r}\iter{k} \eqbydef A{\bf q}\iter{k} - \nu\iter{k}{\bf
q}\iter{k}$ \`e il residuo all'iterazione $k$-esima.
