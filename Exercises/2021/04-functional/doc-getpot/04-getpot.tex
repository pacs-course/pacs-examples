\documentclass[10pt]{beamer}
\usetheme{default}
\setbeamercovered{invisible}
\setbeamertemplate{navigation symbols}{}
\setbeamertemplate{footline}{
    \flushright{\hfill \insertframenumber{}/\inserttotalframenumber}
}

\usepackage{listings}

% User-defined colors.
\definecolor{DarkGreen}{rgb}{0, .5, 0}
\definecolor{DarkBlue}{rgb}{0, 0, .5}
\definecolor{DarkRed}{rgb}{.5, 0, 0}
\definecolor{LightGray}{rgb}{.95, .95, .95}
\definecolor{White}{rgb}{1.0,1.0,1.0}
\definecolor{darkblue}{rgb}{0,0,0.9}
\definecolor{darkred}{rgb}{0.8,0,0}
\definecolor{darkgreen}{rgb}{0.0,0.85,0}

% Settings for listing class.
\lstset{
  language=C++,                        % The default language
  basicstyle=\small\ttfamily,          % The basic style
  backgroundcolor=\color{White},       % Set listing background
  keywordstyle=\color{DarkBlue}\bfseries, % Set keyword style
  commentstyle=\color{DarkGreen}\itshape, % Set comment style
  stringstyle=\color{DarkRed}, % Set string constant style
  extendedchars=true % Allow extended characters
  breaklines=true,
  basewidth={0.5em,0.4em},
  fontadjust=true,
  linewidth=\textwidth,
  breakatwhitespace=true,
  showstringspaces=false,
  lineskip=0ex, %  frame=single
}

\usepackage{verbatim}

\begin{document}
    \title{GetPot}
    \author{Pasquale Claudio Africa}
    \date{19/03/2021}

\begin{frame}[plain, noframenumbering]
    \maketitle
\end{frame}

\begin{frame}{Intro to Getpot}
    \begin{figure}
        \centering
        \includegraphics[width=0.5\textwidth]{images/GetPot_logo.jpg}
    \end{figure}

    \begin{block}{Web page}
        \centering
        \begin{itemize}
            \item \url{http://getpot.sourceforge.net/}
        \end{itemize}
    \end{block}

\end{frame}


\begin{frame}{GetPot}

    GetPot is a \emph{header file only} library,
    to facilitate command line and config file parsing.
    Useful for changing algorithm parameters without recompiling
     etc\ldots\\[1cm]

    GetPot provides a class to parse \texttt{argc} and \texttt{argv}
    in alternative to POSIX standard \texttt{getopt} (in \texttt{C}).

\end{frame}

%---------------------------------------------------------------------------------

\begin{frame}[fragile]{Passing parameters directly on the command line}

    \lstinputlisting{./examples/ex1.cpp}

    Parameters are read via the call \texttt{operator()} for the \texttt{GetPot} class.
    It requires the name of the parameter to be read and the default value.
    The type is deduced from the class of the default value.

\end{frame}

%---------------------------------------------------------------------------------

\begin{frame}[fragile]{Run}

    \begin{verbatim}
./main
./main a=10 b=70
./main b=70.5 n_intervals=100
./main n_intervals=100 a=10
    \end{verbatim}

    \vspace{1cm}
    
   	Sorting of command line arguments is not relevant.

\end{frame}

%---------------------------------------------------------------------------------

\begin{frame}[fragile]{Configuration files}
    \verbatiminput{./examples/data}
\end{frame}

%---------------------------------------------------------------------------------

\begin{frame}[fragile]{How to parse from C++}
    \lstinputlisting{./examples/ex2.cpp}
\end{frame}

%---------------------------------------------------------------------------------

\begin{frame}[fragile]{How to parse from C++}
    \lstinputlisting{./examples/ex3.cpp}
\end{frame}

%---------------------------------------------------------------------------------

\begin{frame}[fragile]{Why not both?}
    The name of the config file may be passed on the command line
        
    \lstinputlisting{./examples/ex4.cpp}
\end{frame}

%---------------------------------------------------------------------------------

\begin{frame}[fragile]{Run}
    \begin{verbatim}
./main -f data1
./main --file data1
./main
    \end{verbatim}
\end{frame}
\end{document}
