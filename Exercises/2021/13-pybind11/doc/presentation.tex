\documentclass[aspectratio=169,11pt]{beamer}
\usetheme{focus}

\definecolor{main}{HTML}{015697}
\definecolor{background}{HTML}{f8fbff}

\usepackage{pifont}
\newcommand{\pro}{\textcolor{example}{\ding{51}}}
\newcommand{\con}{\alert{\ding{55}}}

\usepackage{minted}
\setminted{bgcolor=main!10!background}
\usemintedstyle{manni}

\title{Introduction to pybind11}
\author{Pasquale Africa}
\institute{Politecnico di Milano}
\date{2021-05-28}
\titlegraphic{}

\begin{document}
\begin{frame}[plain]{}
    \maketitle
\end{frame}

\begin{frame}{Introduction}
    \textit{\textbf{\texttt{pybind11}} is a lightweight header-only library that exposes \texttt{C++} types in \texttt{Python} and vice versa, mainly to create \texttt{Python} bindings of existing \texttt{C++} code.}
    \vfill
    \textit{Its goals and syntax are similar to the excellent Boost.Python library by David Abrahams: to minimize boilerplate code in traditional extension modules by inferring type information using compile-time introspection.}
    \vfill
    \textit{The main issue with Boost.Python is that Boost is an enormously large and complex suite of utility libraries that works with almost every \texttt{C++} compiler in existence. This compatibility has its cost: arcane template tricks and workarounds are necessary to support the oldest and buggiest of compiler specimens. Now that \texttt{C++11}-compatible compilers are widely available, this heavy machinery has become an excessively large and unnecessary dependency.}.
    \vfill
    \textbf{Documentation}: \url{https://pybind11.readthedocs.io/en/stable/index.html}.
\end{frame}

\begin{frame}[fragile]{Examples}
    This directory is a clone of \url{https://github.com/tdegeus/pybind11_examples}.
    \vfill
    The examples have been re-adapted and extended. By default all dependencies (\texttt{Python}, \texttt{pybind11}, \texttt{Eigen}, \dots) are included from the \texttt{mk} modules.
    \vfill
    Please remember to run
    \begin{minted}{bash}
module load pybind11 eigen
    \end{minted}
    before starting.
\end{frame}

\begin{frame}[fragile]{How to compile}
    \begin{minted}{bash}
g++ -std=gnu++11 -O3 -shared -fPIC \
    -isystem /path/to/pybind11/include \
    $(python3-config --cflags --ldflags --libs) \
    example.cpp -o example.so
    \end{minted}
    or, better, write a \texttt{CMakeLists.txt} like:
    \begin{minted}{cmake}
cmake_minimum_required(VERSION 2.8.12)
project(example)

find_package(pybind11 REQUIRED)
include_directories(SYSTEM ${pybind11_INCLUDE_DIRS})
pybind11_add_module(example example.cpp)
    \end{minted}
\end{frame}

\begin{frame}[fragile]{How to compile (using CMake) and run}
    Compile:
    \begin{minted}{bash}
cd /path/to/example/
mkdir build && cd build
cmake -Dpybind11_DIR=${mkPybind11Prefix} ..
make
    \end{minted}
    Run:
    \begin{minted}{bash}
cd /path/to/example/
PYTHONPATH+=:$(pwd)/build python3 test.py
# Or:
# export PYTHONPATH+=:$(pwd)/build
# python3 test.py
    \end{minted}
    
    See also \texttt{pybind11\_examples/README.md} for further details.
\end{frame}

\begin{frame}[fragile]{\texttt{setup.py}}
    If a \texttt{setup.py} file is provided then the following commands become available:
    \begin{minted}{bash}
# Build.
python3 setup.py build

# Install.
python3 setup.py install [--user]

# Build and install.
pip install [--user] [--verbose] .
    \end{minted}
    The dynamic module will be installed in your local \texttt{Python} path, \textit{e.g.}:
{
    \small
    \begin{minted}{bash}
export PYTHONPATH+=:${HOME}/.local/lib/python3.8/site-packages/
    \end{minted}
}
\end{frame}
\end{document}
