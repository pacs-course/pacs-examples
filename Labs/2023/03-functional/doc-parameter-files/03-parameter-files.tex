\documentclass[10pt,aspectratio=169]{beamer}
\usetheme{default}
\setbeamercovered{invisible}
\setbeamertemplate{navigation symbols}{}
\setbeamertemplate{footline}{
    \flushright{\hfill \insertframenumber{}/\inserttotalframenumber}
}

\usepackage{listings}
\usepackage{multicol}

% User-defined colors.
\definecolor{DarkGreen}{rgb}{0, .5, 0}
\definecolor{DarkBlue}{rgb}{0, 0, .5}
\definecolor{DarkRed}{rgb}{.5, 0, 0}
\definecolor{LightGray}{rgb}{.95, .95, .95}
\definecolor{White}{rgb}{1.0,1.0,1.0}
\definecolor{darkblue}{rgb}{0,0,0.9}
\definecolor{darkred}{rgb}{0.8,0,0}
\definecolor{darkgreen}{rgb}{0.0,0.85,0}

% Settings for listing class.
\lstset{
  language=C++,                        % The default language
  basicstyle=\small\ttfamily,          % The basic style
  backgroundcolor=\color{White},       % Set listing background
  keywordstyle=\color{DarkBlue}\bfseries, % Set keyword style
  commentstyle=\color{DarkGreen}\itshape, % Set comment style
  stringstyle=\color{DarkRed}, % Set string constant style
  extendedchars=true % Allow extended characters
  breaklines=true,
  basewidth={0.5em,0.4em},
  fontadjust=true,
  linewidth=\textwidth,
  breakatwhitespace=true,
  showstringspaces=false,
  lineskip=0ex, %  frame=single
}

\usepackage{verbatim}

\begin{document}
    \title{GetPot and JSON}
    \author{Matteo Caldana}
    \date{10/03/2023}

\begin{frame}[plain, noframenumbering]
    \maketitle
\end{frame}

\begin{frame}{Intro to Getpot}
    \begin{figure}
        \centering
        \includegraphics[width=0.5\textwidth]{images/GetPot_logo.jpg}
    \end{figure}

    \begin{block}{Web page}
        \centering
        \begin{itemize}
            \item \url{http://getpot.sourceforge.net/}
        \end{itemize}
    \end{block}
\end{frame}

\begin{frame}{GetPot}
    GetPot is a \emph{header file only} library,
    to facilitate command line and config file parsing.
    Useful for changing algorithm parameters without recompiling
     etc\ldots\\[1cm]

    GetPot provides a class to parse \texttt{argc} and \texttt{argv}
    in alternative to POSIX standard \texttt{getopt} (in \texttt{C}).
\end{frame}

\begin{frame}[fragile]{Passing parameters directly on the command line}

    \lstinputlisting{./examples-getpot/ex1.cpp}

    Parameters are read via the call \texttt{operator()} for the \texttt{GetPot} class.
    It requires the name of the parameter to be read and the default value.
    The type is deduced from the class of the default value.
\end{frame}

\begin{frame}[fragile]{Run}
    \begin{verbatim}
./main
./main a=10 b=70
./main b=70.5 n_intervals=100
./main n_intervals=100 a=10
    \end{verbatim}

    \vspace{1cm}
    
   	Sorting of command line arguments is not relevant.
\end{frame}

\begin{frame}[fragile]{Configuration files}
    \verbatiminput{./examples-getpot/data}
\end{frame}

\begin{frame}[fragile]{How to parse from C++}
    \lstinputlisting{./examples-getpot/ex2.cpp}
\end{frame}

\begin{frame}[fragile]{How to parse from C++}
    \lstinputlisting{./examples-getpot/ex3.cpp}
\end{frame}

\begin{frame}[fragile]{Why not both?}
    The name of the config file may be passed on the command line
        
    \lstinputlisting{./examples-getpot/ex4.cpp}
\end{frame}

\begin{frame}[fragile]{Run}
    \begin{verbatim}
./main -f data1
./main --file data1
./main
    \end{verbatim}
\end{frame}



\begin{frame}{JSON}

\href{https://www.json.org/json-en.html}{JSON} (JavaScript Object Notation) is a lightweight data-interchange format. It is easy for humans to read and write. It is easy for machines to parse and generate. It is based on a subset of the JavaScript programming language. Gain traction to exchange data between client and server, nowdays is very widely used and supported by many languages.\\[0.5cm]

JSON object is an unordered set of name/value pairs. An object begins with { (left brace) and ends with } (right brace). Each name is followed by : (colon) and the name/value pairs are separated by , (comma).\\[0.5cm]

A value can be one of the following
\begin{multicols}{2}
\begin{itemize}
\item object itself
\item list
\item string
\item boolean
\item number 
\item \texttt{null}
\end{itemize}
\end{multicols}

\end{frame}

\begin{frame}{JSON}
\url{https://github.com/nlohmann/json} is a C++ library with 
\begin{itemize}
    \item Intuitive syntax: it behaves just like an STL container. In fact, it satisfies the \href{https://en.cppreference.com/w/cpp/named_req/ReversibleContainer}{ReversibleContainer} requirement.
    \item Trivial integration. Our whole code consists of a single header file json.hpp
    \item Memory efficiency. Each JSON object has an overhead of one pointer (the maximal size of a union) and one enumeration element (1 byte). The default generalization uses the following C++ data types: \texttt{std::string} for strings, \texttt{int64\_t}, \texttt{uint64\_t} or double for numbers, \texttt{std::map} for objects, std::vector for arrays, and \texttt{bool} for Booleans.
\end{itemize}

\end{frame}

\begin{frame}[fragile]{A simple example}        
    \lstinputlisting{./examples-json/ex1.cpp}
\end{frame}

\end{document}
